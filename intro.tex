
\section{Introduction}


Chip Multi-Processors (CMPs) have become the mainstream processors for
mobile, desktop and cloud computing platforms due to the power
budget and the limits on clock frequency scaling. While
thread-level parallelism can be well leveraged with CMPs \cite{CMP05}
for throughput oriented applications, the single-thread performance of
a processor remains important to time critical workloads. Therefore,
various ASIC-based accelerators are utilized to improve single-thread performance. 

However, ASIC based on-chip accelerators do not respond well to
emerging dynamic workloads where applications come and go either on
demand or in a unpredictable manner. For example, the user of a smartphone may
install or uninstall Apps at any time, leading to a changing mixture
of applications. A cloud computing platform such as Amazon EC2
\cite{amazon-ec2} executes various workloads such as Web
\cite{Chen:2012jo}, data mining \cite{ec2-datamining}, DNA sequencing
\cite{ec2-dna} submitted from users' Virtual Machines (VMs) at
arbitrary moments. The built-in staic accelerators may not be needed at certain time periods
or for a given workload mix, thus
become dormant for a prolonged time span, wasting silicon resources.

A complementing approach is to use off-chip programmable hardware
accelerators like GPU and FPGA to speed up complex workloads
\cite{GPUFPGA, fpga-acc}. However, such hardware acceleration units do
not meet the stringent time requirements of delay-sensitive workloads
due to conventional system architectures. These units are typically
connected to the processor cores via off-chip interconnects
(e.g. PCIe, QPI \cite{intel-qpi} or HyperTransport
\cite{amd-hypertransport}) with serious latency limits, resulting
significant delays in data transfer. Thus, even with massively
parallel processing capabilities, FPGAs and GPUs are challenged to
deliver the expected speedup in reality.  
%Their performance advantages
%and flexibility in reprogramming cannot be harnessed to the full
%extent until an architectural change occurs on their interconnects to
%cores.

Leveraging on-chip accelerators in existing and new applications is
not trivial as redevelopment cost is prohibitive.  
%Allowing an applications' execution flow to be transparently directed to accelerators
%without any user intervention remains an interesting but unsolved
%issue. 
As new instructions are typically added along with SoC
accelerators, rewriting and recompiling an application's code is
mandatory. Fine-grained accelerations such as
\cite{Govindaraju:2012fn} also require compiler support. Since IPs for
FPGAs and programs developed for GPUs keep increasing, it makes sense
to incorporate them through coarse-grained acceleration, i.e. the
speedup of an entire function (e.g. 3DES encryption) as an element of
an existing software library (e.g. libopenssl). Such library based
approach facilitates acceleration without incurring (re)development
costs. 

The power efficiency \cite{hamada09,thomas09} and performance benefits
of recent programmable hardware have been driving the momentum of
heterogeneous computing platforms that combine general-purpose cores
and reconfigurable logics (e.g. Convey HC-1 \cite{brewer09} and Cray
XD1\cite{Ulmer:2005vh}) for high performance computing (HPC)
applications. However, the workloads in many computing environments
are far more dynamic and versatile than those in HPC domain. The
fusion of general-purpose cores and programmable logics for
power-efficient computing is inadequately investigated in terms of
architecture trade-offs on performance and power, transparent
acceleration, and accelerator-aware scheduling, all of which are
critical to the practical deployment of programmable heterogeneous
architectures.

We propose {\em Transformer}, a heterogeneous architecture with both
general purpose cores and on-chip programmable accelerator logics for
addressing the challenges introduced by dynamic workloads in many
emerging scenarios. This architecture differs from existing works in
several aspects: (1) the architecture consists of on-chip programmable
accelerators with general purpose cores.  Sharing the memory hierarchy
with cores, the accelerators are promoted to first-class citizens and
access cores' data in an efficient way to reduce latency;  
(2) sharing an accelerator unit among multiple acceleration functions
improves the utilization of the chip resources. Optimal area resource
allocation to cores and accelerators leads to improved performance and
energy efficiency. Our proposed heuristics maximize the memory
bandwidth utilization and improve speedup of unpredictable workload
mixtures under given resource constraints; and (3) the architecture
enables transparent acceleration with novel middleware support,
significantly reducing deployment costs.  It is worth noting that,
while the programmable accelerator logics in the proposed architecture
can be GPU based coprocessors \cite{intel-gpu}, we focus in this paper
on FPGA-style reconfigurable logics due to the large set of existing IP
designs available for FPGAs and their improved power
efficiency. Nevertheless the middleware and scheduling algorithms
studied in this research are generic enough for benefiting either
implementation.

We make the following contributions in this work:
\begin{enumerate}

\item We present a deployable heterogeneous architecture with run-time
  programmable on-chip accelerators. Our study evaluates the
  performance benefits of on-demand accelerators, and provides insight into how different parameters affect the performance and power efficiency on the heterogeneous architecture.

\item We design a suite of middleware and scheduling algorithms for
  supporting transparent acceleration, which is an enabling factor for
  the wide deployment of accelerators, but not specifically addressed in
  prior research. To the best of our knowledge, our work is the first
  to address this issue and present a viable solution.

\item We characterize the power consumption of the proposed
  heterogeneous architecture with industrial level power modeling
  tools. This study gives important insights on the scalability and
  power-performance trade-offs.

\end{enumerate}

The paper is organized as follows. Section \ref{sec_related} discusses
prior work related to our research. Section \ref{sec_arch} describes
the proposed programmable heterogeneous architecture, followed by the
transparent acceleration mechanisms detailed in Section
\ref{sec_transacc}. We present the accelerator-aware scheduling and
accelerator combination algorithms in Section
\ref{sec_runtime_reconfig}. Performance evaluation methodology and
experiment results are presented in Section \ref{sec_perf}. Finally we
conclude the paper in Section \ref{sec_concl}.

 \if 0 FPGA as a
co-processing unit, has demonstrated the ability to speed up a variety
of applications, such as image processing \cite{imageacc}, data mining
\cite{data-mining-ref}, bioinformatics \cite{bioacc1} \cite{bioacc2},
navigation \cite{naviacc} and encryption/decryption
\cite{encryptionacc}. As another popular alternative acceleration
method, GPGPUs are less expensive and have higher memory bandwidth and
a larger number of programmable cores with thousands of hardware
threads than FPGA. So if the data to be processed can be simply
divided into many parallel trunks without any dependency or shared
data, it can be easily processed though the GPU
optimizations. However, there are several limitations of GPU that may
severely affect the computing performance on threaded GPU
platform. First, it is usually difficult to find how much data
parallelism lies in the application because no aspects of GPUs are
transparent to programmers \cite{microsoft06}. Second, scatter and
gather are two basic operations performed by GPU, which will also
introduce a gigantic amount of memory access latency and degradation
of memory bandwidth because of their access randomness
\cite{GPUlimit1}. Third, there are many domains of application with
large data dependency that are not optimized well on GPU
\cite{GPUlimit2}.  \fi
