\section{Related Work}
\label{sec_related}
Cong et al. \cite{accrich,cong-islped12,cong-saw11} propose a
heterogeneous architecture called CHARM consisting of loosely-coupled on-chip
accelerators. By composing the accelerator building blocks, the
architecture is able to dynamically speed up medical imaging benchmarks by a factor of
to 3.7$\times$ while reducing energy consumption by a factor of 4.7$\times$.  In
spite of its effectiveness in terms of computation acceleration, this
accelerator-rich architecture introduces new accelerator instructions,
requires recompiling the applications, and in turn lacks run-time
reconfigurability.  Garcia et al. propose software based kernel sharing in a
multiprocessor system with reconfigurable hardware
	\cite{Garcia:2008iy}. This concept is similar to CHARM in terms of the emphasis upon strategies that best utilize the existing
accelerator.

Our work differs from CHARM \cite{accrich,cong-islped12,cong-saw11} in several ways.
First, the ``{\em Transformer}'' is targeted at cloud and mobile applications
where the execution environment has to be able to cope with the dynamics of
unpredictable workloads that can arrive or depart at any time. The proposed solution
 focuses on profiling the workloads at run-time and dynamically reprogramming
the acceleration functions. In contrast, CHARM 
emphasizes sharing the existing accelerators amongst the software
threads, and lacks the ability to cope with emerging workloads at
run-time. Second, in order to improve area resource utilization, the ``{\em Transformer}'' incorporates 
centrally located reconfigurable logic, instead of distributed fine-grained accelerator blocks
\cite{accrich}. Thanks to the partial reconfiguration technology, the chip
resources dedicated to the acceleration logic can be re-programmed and
shared by multiple accelerator engines for different functions.
In order to maximize memory bandwidth utilization as well as the improvements in terms of speed,
we develop heuristics that combine accelerators on-demand.
Third, the ``{\em Transformer}'' has a set of
middleware innovations that avoid rewriting user applications with a
library-oriented coarse-grained approach. This is supported by the
growing set of available soft IP cores for reconfigurable logic
\cite{opencores,design-reuse,free-ip}. By comparison, CHARM requires
rewriting user code.

Govindaraju et al. propose {\em DySER}, which contains specialization in terms of both functionality and
data parallelism \cite{Govindaraju:2012fn,Govindaraju:HPCA11}. DySER
provides a feasibile solution that dynamically specializes execution
resources and creates various data paths for a parallelizable
``hot-spot'' in the workload. The net result leads to improvements in terms of performance and energy
efficiency. By taking a different approach, the {\em Transformer} aims at developing
coarse-grained acceleration at the level of a library function. On the one hand,
{\em DySER} relies on the compiler to partition hot spots consisting of compute and
data subregions, which can then be offloaded to dynamically formed functional
units. On the other hand, the {\em Transformer} is able to accelerate closed-source executables independently of the compiler, and contains
the necessary hardware and middleware to perform run-time reconfiguration to avoid
recompilation.

Both {\em Garp} \cite{Garp:1997,Garp:2000} and the {\em Transformer} aim to improve computation performance and power
efficiency through FPGA-based parallelization. However, the two
architectures focus on different levels of granularity in terms of parallelism. Similar
to {\em DySER}, {\em Garp} targets finer granularity by
optimizing basic operations in order to achieve instruction level
parallelism. This form of parallelization is heavily reliant upon the compiler given the necessity of
precisely identifying potentially parallelizable code and the mapping of that code
to an acceleration block in a way that leads to performance improvements.
At the instruction level, the penalty of imprecise identification can be significant.
%Otherwise, the compiler optimized code might compromise the
%performance. 
Our approach focuses on performing optimization at the library level.
The probability of performance improvements increases by considering the issues of data sharing and computation from a global
perspective. 
An optimized version of a function on an FPGA can outperform the CPU version of the same
function by a factor of ten or even one-hundred in terms of speedup. The only requirement is to 
plug in the right accelerator at the right time to tackle the
the dynamics of the workload. Profiling and scheduling methods can also be applied with respect to fine-granularity acceleration.

Other works such as HiPPAI \cite{Stillwell:2009if} and EXOCHI
\cite{Wang:2007bc} propose new programming environments or interfaces
for hardware accelerated SoC platforms. HiPPAI abstracts a layer of the
accelerator interface within the OS in order to schedule tasks either to the
accelerator or to the general purpose core. However, this layer of
abstraction lacks the awareness of the run-time dynamics within the
accelerator. By contrast, we propose a reconfiguration controller that
 tracks usage information within the logic, thereby using 
the logic resources efficiently without the need for reprogramming. 

With respect to high performance big data applications, LINQits is proposed as
a flexible hardware template that can be mapped onto programmable
logic or ASICs in a heterogeneous system-on-chip for mobile devices
or servers \cite{Chung:2013:LBD:2485922.2485945}. LINQits is designed to accelerate a domain-specific query
language called LINQ.  As a result, LINQits requires that applications either be written or rewritten with
LINQ in order to take advantage of the hardware acceleration. 

While the LINQits architecture has similarities with our proposed {\em
  Transformer} architecture, taking on a ``core + FPGA'' form, there exist
notable differences between them. First, the {\em Transformer} is a
generic architecture not tied to a particular application domain
whereas LINQits is designed and optimized for big data applications.
Second, no rewriting or recompilation of applications is required with the {\em Transformer}.  
By constrast, LINQits requires that applications be written with LINQ language
constructs, which are not applicable to non-database applications.

Huang et al. propose an aggregate gain algorithm (AG)
\cite{Huang:2009hs} in order to predict the future acceleration requirements of the
workload. This prediction mechanism may work properly within a subset of specific
 computing domains, however, cloud workloads are too dynamic to be
accurately predicted. Furthermore, Huang's work does not consider cache coherency, while the
number of LUTs serves as the only resource constraint. By comparison, the {\em Transformer}
maintains coherency in terms of the cache for each core as well as the local memory for each accelerator.
In addition, the {\em Transformer} considers the constraints of all the logic resources, which includes 
but is not limited to the LUT, the BRAM, as well as the SLICE.


Mignolet et al. \cite{Mignolet:2003gr} introduce a feasible way of
relocating tasks between the software path and the accelerated hardware
path. They implement a new OS (OS4RS) which contains a hardware
abstraction layer and a communication interface to enable OS task
scheduling between software and hardware. 
However, the {\em Transformer} does not require changes to the OS
layer. Instead, transparent acceleration in the {\em Transformer} is
achieved by way of a wrapper library which resides at the middleware layer,
thus improving the portability of the design.


Supercomputers pursue the highest computation performance
\cite{Ulmer:2005vh} through the support of FPGAs. For instance, each node
 out of the six nodes within a Cray XD1 chassis integrates one Xilinx Virtex-4 FPGA
connected via a RadipArray Interconnect to four memory banks shared
between two AMD Opteron processors. Though both provide run-time
reprogrammability, Cray XD1 differs from the {\em Transformer} in two
aspects.
First, the reconfigurable logic on the Cray XD1 consists of an off-chip
coprocessor which communicates with the CPUs under the control of Rapid
Array Processors whereas the {\em Transformer} integrates the reconfigurable
logic onto the SoC, permitting on-chip data sharing between the cores.
Second, in order to utilize the coprocessors located on the Cray XD1, users need to incorporate the
vendor-provided FPGA Linux API into their software design. In
contrast, the {\em Transformer} provides a transparent scheme for invoking
the accelerators.

As an ``extreme" form of fixed specialization, conservation cores
\cite{Venkatesh:2010:CCR:1735970.1736044,Venkatesh:2010:CCR:1736020.1736044,Venkatesh:2010:CCR:1735971.1736044},
or c-cores, are specialized processors that focus only on energy and
energy-delay efficiency rather than improving performance. C-cores
provide a promising reduction in terms of energy consumption, up to
16.0$\times$ for specific functions and up to 2.1$\times$ for the whole
application, at the risk of compromising performance.  The {\em Transformer} produces a comparable reduction in energy
consumption by a factor of up to 6.9$\times$ without the risk of compromising performance. 
However, we argue that QoS is one of the most important
concerns for most cloud service providers, specifically in terms of providing
the best user experience. Thus, the balance between performance and
power consumption should be placed under careful consideration. The {\em Transformer}
provides a solution that considers both the performance and the power aspects
for cloud services. In addition, the service provider typically does
not have access to the source code used or owned by the customers within a
cloud computing environment. As a result, c-cores become ineffective within such scenarios. The {\em Transformer}'s middleware approach, based on a wrapper library, imposes no requirements on source code access, and therefore is feasible for a wide range of applications. 

Given the reprogramming and partial-reconfiguration costs for FPGA-type
logic, ranging in the order of tens to hundreds of milliseconds depending on
the size of the bit stream \cite{Liu:2009ie}), the benefits of adapting
the accelerator functions at run-time outweigh the previously stated
costs. Characterization of cloud computing workloads has demonstrated that the lifetime of various
applications ranges from 300 seconds to 86400 seconds \cite{CloudWorkload}, implying that an
accelerated function would be used for a considerable duration of time. The preceding findings give impetus
for reprogramming accelerators with different
functions on-demand.

Our proposed architecture is built on top of partial reconfiguration
technology, such as the XILINX Dynamic Partial Reconfiguration (DPR) technology
\cite{PRUserGuide}, in order to reduce the reconfiguration time without
interrupting the operation of other logic. For example, in the Xilinx DPR
design process, a library of hardware modules, called Partially
Reconfigurable Modules (PRMs), and their corresponding Partial Bit
Streams (PBSs) are created in advance. 
%The circuits that are not configurable during runtime are called Static Region. 
During runtime, PBSs are configured to a region on the device referred to as the
Partial Reconfigurable Region (PRR) through a primitive configuration
module known as the Internal Configuration Access Port (ICAP)
\cite{Hansen:2011dt,Liu:2009ie,McDonald:2008ec}. We leverage the ICAP-like
structure to reprogram the configurable logic on demand.



