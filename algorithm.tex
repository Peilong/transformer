\section{Run-time Reconfiguration }
\label{sec_runtime_reconfig}

\subsection{Overview}
\label{subsec_runtime_reconfig_overview}

Run-time reconfiguration refers to the decisions made by the controller
to accelerate certain software libraries, 
as determined by the run-time workload characteristics, as well as how to effectively assess the
workloads periodically. The heart of reprogramming hardware
accelerators at run-time lies in the algorithms as well as the mechanisms employed to
identify software functions that should be accelerated, keeping in mind the goals of sharing
the memory and logic resources as well as maximizing the resource utilization of the
reconfigurable logic. Since reprogramming the accelerator logic 
introduces a delay, we do not want the aforementioned overhead to outweigh the benefits of accelerating time-consuming
tasks. We propose a reconfiguration controller which addresses three
questions.
First, what it the most effective way to track the demands of the candidate workloads? 
Second, when should the acceleration logic be reprogrammed?
Third, which accelerator or combination of accelerators should be instantiated?


The aforementioned questions are answered in two steps:
demand tracking and request scheduling. As described in
Section \ref{sec_transacc}, the first step is to intercept all of the
library calls with wrapper functions and keep track of the demands
using a table of counters called Request Counters (RC). The RCs
are regarded as a request log that contains updated information for the most
recent time window. In the second step, heuristic algorithms are applied in order to
schedule the requested function within the hardware accelerators. 
The corresponding algorithms are described in Section \ref{subsec_combo}. 

\subsection{Demand Tracking} 

The demands of computing a candidate function are tracked by the
wrapper function library. The library maintains a table of candidate
hardware accelerators along with the corresponding functions they are designed to speed up. Each call
to a function is intercepted and recorded by the wrapper function, which is then used as an
index into the table in order to increment the corresponding request counter. The
counters serve as input metrics to the scheduling algorithm, which determines the appropriate
functions to accelerate. Tracking the demands only requires performing two memory operations, a table
lookup, a read, followed by a table update, a write.
The overhead of maintaining the counters is trivial, even when the number
of accelerators scales into the hundreds. Specifically, each counter
update is only performed once for each library function call.  By comparison, the library function itself is typically time-consuming, in the order millions of cycles. Moreover, the counters are reset periodically in order to record the most recent demands. 


In the presence of dynamic workloads and heterogeneous processing
elements, which include general-purpose cores as well as programmable accelerators, there
could be multiple, time-dependent candidate functions to accelerate.
 The acceleration of the workload turns into an optimization
problem where the objective is to schedule tasks on multiple
processors in order to minimize the execution time. Given that Hochbaum et
al. have proven similar scheduling on uniform processors is NP-hard
\cite{hochbaum88}, the scheduling on heterogeneous processing 
elements must also be NP-hard. Therefore, we derive heuristics to address
the remaining two questions.
 Specifically, we address the second question with
the heuristics for determining the reconfiguration timing in
Section \ref{subsec_ranking}, and the third question with the heuristics for
scheduling and combining accelerators in Section \ref{subsec_combo}.

\subsection{Request Ranking}
\label{subsec_ranking}

Each time function acceleration requests are detected by the
wrapper library, we increment the corresponding request counter in a
time window of length $T$ before resetting it in the next window. 
Not only are we interested in the number of calls within a time window $T$, but also the {\em changes} in the demands
between subsequent time windows. Specifically, the dynamic workloads
could experience demand fluctuation, implying that the historical trend in the
past time windows is just as important as the number of calls in one or
more time windows.
Smaller values for $T$ lead to frequent counter resets and trace the trend
of demands with finer granularity, while larger values for $T$ avoid
oversampling the demands as well as unnecessarily frequent reprogramming of the accelerators. Therefore, the value of $T$ should be chosen
appropriately such that it does not add significant overhead when
comparing the execution times of the applications under study, while being sufficient
to capture the details of the changes in demand for the acceleration requests. 

We maintain an array of counters $C(x, i)$ collected in the past $R$
time windows, where $x$ represents the corresponding accelerator
function and $1<i<R$. At run time, there exists an $f \times R$
matrix $M$ that reflects the demands for $f$ accelerator functions in
the past $R$ time windows.  After obtaining $M$, we use the following
two metrics to evaluate the total demand $D_x$ and the rate of demand
changes $DC_x$ for every function $x$: $D_{x} = \sum_{i=1}^{H}C(x,i)$
and $DC_{x} = \sum_{i=1}^{H-1}(|C(x, i+1)-C(x, i)|)$.  The priority
$P_x$ of function $x$ is then calculated as $P_x = a \times D_x +
(1-a) \times DC_x$. We set $a$ to be 0.5 in our implementation given that we
regard the changes in demand to be as important as the total number of demands. $P_x$
is used to rank the accelerators requests in descending
order. Changes that occur in the ranking from the past time window trigger
reprogramming the accelerator functions in the on-chip logic. We regard the aforementioned scheduling strategy as the baseline and denote it as ``na\"{\i}ve'' in the performance evaluation.

\subsection{Combining Accelerators}
\label{subsec_combo}



Combining acceleration functions to maximize the utilization of reconfigurable logic
allows more effective speedup and power savings.

As each accelerator function consumes a certain amount of
on-chip resources such as LUT, memory, and data access
bandwidth to achieve a particular speedup factor, as shown in Table
\ref{tbl_benchmark}, the choice of accelerators to maximize gains
under resource constraint is a combinatorial optimization problem, i.e., knapsack problem. We will discuss two variants of knapsack problem
applicable in our accelerator scheduling algorithm: one is 0-1
knapsack problem, i.e., an acceleration function is either chosen or not
chosen;  the other is bounded knapsack problem: up to $n$ copies of
the same acceleration function can be instantiated as independent
engines into the accelerator. With the total resource available to the programmable accelerator as
a constraint, we consider two different prioritization strategies to minimize the
execution time: memory bandwidth utilization first and acceleration
speedup first when solving the knapsack problems.

\subsubsection{Mathematical Model}

We model an accelerator function as $A(SLICE, DSP, BRAM, Bandwidth, Speedup)$,
where {\em SLICE, DSP, BRAM} are logic, application-specific blocks and
memory cells resources, respectively, consumed by the function when it is
instantiated in the reconfigurable logic unit. {\em Bandwidth} is the
requirement of the function on data bus in order to keep the unit
busy. {\em Speedup} is the effective speedup of an accelerated function on the
reconfigurable logic compared to its software implementation. 

\begin{table*}[ht]
\scriptsize
\centering
\begin{center}
\begin{tabular}{|c|c|c|c|c|c|c|c|c|c|c|c|}
\hline 
\textbf{Accelerators}& \textbf{SLICE} & \textbf{SLICE \%} & \textbf{FF} & \textbf{FF \%} & \textbf{LUT} & \textbf{LUT \%} & \textbf{BRAM} & \textbf{BRAM \%} & \textbf{Bandwidth MB/s} & \textbf{Speedup} & \textbf{Power W} \\ 
\hline
\hline  
3DES          & 1148  & 24.7  & 807  & 8.7  & 1081 & 11.6  & 8    & 40   & 392   & 25.2     & 0.157\\ 
\hline 
IDSI          & 1637  & 35.2  & 530  & 5.7  & 891  & 9.6   & 0    & 0    & 970   & 13.1     & 0.235\\ 
\hline 
SLAM-C        & 812   & 17.4  & 973  & 10.4 & 854  & 9.2   & 0    & 0    & 479   & 33.0     & 0.156\\ 
\hline 
SLAM-J        & 983   & 21.1  & 1027 & 11.1 & 872  & 9.4   & 0    & 0    & 458   & 29.0     & 0.147\\ 
\hline 
SURF          & 1877  & 40.3  & 1163 & 12.5 & 705  & 7.6   & 5    & 25   & 983   & 25.0     & 0.161\\ 
\hline 
Segmentation  & 2918  & 62.7  & 930  & 10.0 & 630  & 6.8   & 12   & 60   & 848   & 45.5     & 0.178\\ 
\hline 
SmithWaterman & 626   & 13.4  & 1285 & 13.8 & 1034 & 11.1  & 20   & 100  & 403   & 36.6     & 0.182\\ 
\hline 
Jacobi        & 1201  & 25.8  & 1431 & 15.4 & 1218 & 13.1  & 2    & 10   & 1112  & 30.6     & 0.153\\ 
\hline 

\end{tabular} 
\caption{Benchmark accelerator logic cost, bandwidth, speedup and
  power {\em (Use Xilinx SPARTAN 3E as a reference)}} 
\label{tbl_benchmark}
\end{center}
\end{table*}

Table \ref{tbl_benchmark} summarizes the library functions under
study. These functions (explained in detail in \cite{accstore}) are
representative time-consuming workloads in a wide range of application
domains.  In this table, all data (except bandwidth consumption and
speedup) are derived from the open source synthesizable accelerator
store \cite{accstore}. We measure their memory bandwidth with Intel
VTune Amplifier XE \cite{vtune}. The speedup of each accelerator is
calculated by dividing execution time of one iteration of the software
benchmark on a 1.6 GHz single core by its corresponding hardware design
on a 200 MHz FPGA. {\em Transformer} applies Xilinx Spartan-3E XC3S500E
FPGA \cite{spartan3e} as a reference on-chip accelerator, which has a
comparable number of transistors ({$100\sim300$ million (500K gates)})
as eight Atom 330 cores ({$\sim 47$ million per core})
\cite{atom-spec}. The total number of SLICE on chip is 4656, LUT is
9312, FF is 9312, DSP48 is 768 and BRAM is 20. We also show the
percentage of resource consumption of each acceleration function. It
can be observed that the numbers of SLICEs and BRAMs are two major
resource constraints. Therefore, we reduce the multi-dimensional
knapsack problem to a two-dimension problem. That is, we consider the
percentage of SLICEs and BRAMs of each benchmark as the two weights in
the knapsack model. The memory bandwidth utilization and speedup are
regarded as two objective functions for reducing the total execution
time of memory-bound and compute-bound applications, respectively.

In the memory bandwidth-first combination, we attempt to maximize the
total bandwidth $\sum_{i=0}^{n}(BW_i \times Acc_i) $ subject to
$\sum_{i=0}^{n}(w_i \times Acc_i) $, where $Acc_i$ $i \in {1,2,3,
  ... ,n}$ denotes different accelerator functions, $BW_i$ being the
bandwidth cost and $w_i$ being their weight, i.e. the SLICE resource
demand. In the speedup-first combination selection, we focus our
efforts on maximize the speedup metrics $\sum_{i=0}^{n}(SP_i) $, where
$SP_i , i \in {1,2,3, \ldots, n}$ denotes the speedup of each
accelerator. We attempt to maximize the two metrics because improving
the system bandwidth utilization and choosing a function with
significant speedup are two main approaches to improve the system
performance. For example, our experiment results shows that if we
offload memory-bound benchmarks such as IDSI, SURF, Segmentation and
Jacobi to reconfigurable logics with higher memory bandwidth, we can
expect better overall performance and power efficiency. These
heuristics are shown to be effective through extensive experiments
described in Section \ref{sec_perf}.

\subsubsection{Dynamic Programming}

Both two-constraint 0-1 knapsack problem and two-constraint bounded knapsack problem can be solved
by dynamic programming. 

\noindent{\em 0-1 Knapsack Problem Solution}

Assume we have $n$ acceleration functions to choose from, which are
annotated as $Acc_1$, $Acc_2$, \ldots, $Acc_n$. All $n$ items have
their two-constraint weight $a_i$, $b_i$ and value $v_i$, $i \in [1,
  2, \ldots, n]$. Then we define $V[i, a, b]$ to be the current
maximum value we can obtain with a weight less than or equal to $a$
and $b$ using first $i$ accelerators. The upper limit of two weights
are $A$ and $B$. We derive recursive equations for $V[i, a, b]$:

\begin{equation}
V[i, a, b] = V[i-1, a, b] \quad \text{if} \quad a_i > a, b_i > b,
\end{equation}
meaning if adding a new accelerator exceeds the current
two-dimensional weight limit, then we shall not choose the new
accelerator and the total value will not change, and
\begin{equation}
\begin{array}{c}
	V[i, a, b] = max(V[i-1, a, b], V[i-1, a - a_i, b - b_i] + V[i, a, b])\\
	\text{if} \quad a_i \leq a \quad b_i \leq b.
\end{array}
\end{equation}

Algorithm \ref{alg:0-1-knapsack} describes how we solve two-constraint 0-1
knapsack problem to select a set of accelerator functions. 

\begin{algorithm}[htb]
\scriptsize
\caption{ 0-1 Knapsack Problem Solution}
\label{alg:0-1-knapsack}
\begin{algorithmic}[1] 
\FOR {each $a \in [0, A]$}
	\FOR {each $b \in [0, B]$}
    	\STATE $ V[a, b] = 0; $
    \ENDFOR
\ENDFOR
\FOR {each $i \in [1, n]$ } 
    \FOR {each $a \in [A, a_i]$ }
    	\FOR {each $b \in [B, b_i]$}
    	
        	\IF {$j \geq w[i]$}
            	\STATE $V[i, a, b] = max(V[i-1, a, b], V[i-1, a - a_i, b - b_i] + v[i, a, b]) $
        	\ELSE
            	\STATE $V[i, a, b] = V[i-1, a, b]$
        	\ENDIF
        \ENDFOR
    \ENDFOR
\ENDFOR
\end{algorithmic}
\end{algorithm}

\noindent{\em Bounded Knapsack Problem}

The two-constraint bounded knapsack problem can be solved in a similar
way as 0-1 knapsack problem. Using two extra annotations, $C$ as the
maximum number of acceleration function we can request for a certain
function, and $k_i$ as the number of functions $i$ we choose, we can
derive a similar recursive equation for $V[i, w]$.

\begin{equation}
V[i, a, b] = V[i-1, a, b] \quad \text{if} \quad a_i > a, b_i > b
\end{equation}

and

\begin{equation}
	\begin{array}{c}
		V[i, a, b] = max(V[i-1, a, b], V[i-1, a - k_i\times a_i, b - k_i\times b_i] \\
		+ k_i\times V[i, a, b])\\
		\text{if} \quad a_i \leq a \quad b_i \leq b
	\end{array}
\end{equation}

The algorithm for bounded knapsack problem is similar to Algorithm
\ref{alg:0-1-knapsack}, thus not elaborated due to space limit.



