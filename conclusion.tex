\section{Conclusion}
\label{sec_concl}

In this paper, we propose a heterogeneous architecture, the {\em
  Transformer}, consisting of on-chip, partially reconfigurable logic in order to
transparently accelerate dynamic, unpredictable workloads.
Middleware as well as the reconfiguration controller are introduced in order to trace the
run-time demands of the accelerator functions and invoke the acceleration
path. We derive two scheduling algorithms based on solving the Knapsack
problem in order to combine accelerators. We model the details of the
accelerators in a modified Multi2sim simulator and evaluate the
performance and the power efficiency of the {\em Transformer} with synthetic
workloads from a wide range of domains. The resulting performance and power
efficiency gains are shown to be significant. We study how the 
architectural parameters impact system performance and power. We
investigate the issues of chip area allocation with respect to the cores as well as the accelerators, and make notable observations 
regarding the optimal partitioning of the resources. 

In the near future, we plan to explore finer granularity
reconfigurable resources and accelerator-aware NoC to maximize the
performance of the {\em Transformer}. We also plan to compare the {\em Transformer}
with ASIC-based accelerators, which have significantly greater demands in terms of their VLSI
implementations. Furthermore, an FPGA-based prototype running real-world workloads obtained from cloud
servers is currently in development.

%In this paper, we propose a hybrid architecture with both general
%purpose cores and reconfigurable logic accelerators for power
%efficient computing, which is critical in both cloud computing and
%mobile devices. We also outline a control scheme to reconfigure the
%accelerator at run time in response to the dynamics of the
%workloads. We simulate such architecture with Simics based full system
%simulator by porting benchmarks and designing a device driver for
%accessing the accelerator from user-level threads. Our experimentation
%results show the effectiveness of the reconfigurable logic based
%accelerator and the run-time reconfiguration. This 
%preliminary study encourages us to investigate the performance of the proposed hybrid
%architecture in virtualization environments. 
%%We also plan to optimize the run-time
%%reconfiguration schemes for higher speedup, for example, considering
%%not only the demand of a function, but also the speedup of
%%the accelerated functions.
