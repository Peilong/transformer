\section{Conclusion}
\label{sec_concl}

In this paper, we propose a heterogeneous architecture {\em
  Transformer} with an on-chip, partial reconfigurable logic to
transparently accelerate dynamic unpredictable workloads.
Reconfiguration controller and middleware are introduced to trace the
run-time demands of the accelerator functions and invoke acceleration
path. We derive two scheduling algorithms based on solving Knapsack
problem to combine accelerators. We model the details of
accelerators in a modified Multi2sim simulator and evaluate the
performance and power efficiency of {\em Transformer} with synthetic
workloads from a wide range of domains. The performance and power
efficiency gains are shown to be significant. We study how the 
architectural parameters impact the system performance and power. We
investigate the chip area allocation to cores and accelerators, and make interesting observations on the optimal partition of resources. 

In the near future, we plan to explore finer granularity
of reconfigurable resources and accelerator-aware NoC to maximize the
performance of {\em Transformer}. We also plan to compare Transformer
with ASIC-based accelerators which require much effort in VLSI
implementations. A FPGA-based prototype (with soft cores and partial
reconfiguration) running real-world workloads obtained from cloud
servers is in the works.

%In this paper, we propose a hybrid architecture with both general
%purpose cores and reconfigurable logic accelerators for power
%efficient computing, which is critical in both cloud computing and
%mobile devices. We also outline a control scheme to reconfigure the
%accelerator at run time in response to the dynamics of the
%workloads. We simulate such architecture with Simics based full system
%simulator by porting benchmarks and designing a device driver for
%accessing the accelerator from user-level threads. Our experimentation
%results show the effectiveness of the reconfigurable logic based
%accelerator and the run-time reconfiguration. This 
%preliminary study encourages us to investigate the performance of the proposed hybrid
%architecture in virtualization environments. 
%%We also plan to optimize the run-time
%%reconfiguration schemes for higher speedup, for example, considering
%%not only the demand of a function, but also the speedup of
%%the accelerated functions.
