\subsection{Combining Accelerators}
\label{subsec_combo}



Combining acceleration functions to maximize the utilization of reconfigurable logic
allows for more effective speedup and power savings.

As each accelerator function consumes a certain amount of
on-chip resources such as the LUT, the memory, and the data access
bandwidth to achieve a particular speedup factor, as shown in Table
\ref{tbl_benchmark}, the choice of accelerators to maximize gains
under resource constraint is a combinatorial optimization problem, specifically the knapsack problem. 
We will discuss two variants to the knapsack problem
applicable in our accelerator scheduling algorithm.
The first is a 0-1 knapsack problem, specifically an acceleration function is either chosen or not
chosen.
The second is a bounded knapsack problem, where up to $n$ copies of
the same acceleration function can be instantiated as independent
engines into the accelerator. 
With the total resources available to the programmable accelerator as
one constraint, we consider two different prioritization strategies to minimize the
execution time, specifically the memory bandwidth utilization first strategy as well as the acceleration
speedup first strategy when solving the knapsack problems.

\subsubsection{Mathematical Model}

We model an accelerator function as $A(SLICE, DSP, BRAM, Bandwidth, Speedup)$,
where {\em SLICE, DSP, BRAM} are logic, application-specific blocks and
memory cell resources, respectively.  The aforementioned elements are consumed by the function when it is
instantiated in the reconfigurable logic unit. In the interest of minimizing the amount of time during which the unit remains idle, one
 requirement of the functions is to maximize the {\em bandwidth} of the data bus. {\em Speedup} is the effective speedup of an accelerated function on the reconfigurable logic compared to its software implementation. 

\begin{table*}[ht]
\scriptsize
\centering
\begin{center}
\begin{tabular}{|c|c|c|c|c|c|c|c|c|c|c|c|}
\hline 
\textbf{Accelerators}& \textbf{SLICE} & \textbf{SLICE \%} & \textbf{FF} & \textbf{FF \%} & \textbf{LUT} & \textbf{LUT \%} & \textbf{BRAM} & \textbf{BRAM \%} & \textbf{Bandwidth MB/s} & \textbf{Speedup} & \textbf{Power W} \\ 
\hline
\hline  
3DES          & 1148  & 24.7  & 807  & 8.7  & 1081 & 11.6  & 8    & 40   & 392   & 25.2     & 0.157\\ 
\hline 
IDSI          & 1637  & 35.2  & 530  & 5.7  & 891  & 9.6   & 0    & 0    & 970   & 13.1     & 0.235\\ 
\hline 
SLAM-C        & 812   & 17.4  & 973  & 10.4 & 854  & 9.2   & 0    & 0    & 479   & 33.0     & 0.156\\ 
\hline 
SLAM-J        & 983   & 21.1  & 1027 & 11.1 & 872  & 9.4   & 0    & 0    & 458   & 29.0     & 0.147\\ 
\hline 
SURF          & 1877  & 40.3  & 1163 & 12.5 & 705  & 7.6   & 5    & 25   & 983   & 25.0     & 0.161\\ 
\hline 
Segmentation  & 2918  & 62.7  & 930  & 10.0 & 630  & 6.8   & 12   & 60   & 848   & 45.5     & 0.178\\ 
\hline 
SmithWaterman & 626   & 13.4  & 1285 & 13.8 & 1034 & 11.1  & 20   & 100  & 403   & 36.6     & 0.182\\ 
\hline 
Jacobi        & 1201  & 25.8  & 1431 & 15.4 & 1218 & 13.1  & 2    & 10   & 1112  & 30.6     & 0.153\\ 
\hline 

\end{tabular} 
\caption{Benchmark accelerator logic cost, bandwidth, speedup and
  power {\em (Use Xilinx SPARTAN 3E as a reference)}} 
\label{tbl_benchmark}
\end{center}
\end{table*}

Table \ref{tbl_benchmark} summarizes the library functions under
study. The aforementioned library functions, explained in detail in \cite{accstore}, are
representative of time-consuming workloads relevant to a wide range of application
domains.  In the table, all data, excluding bandwidth consumption and
speedup, are derived from the open source synthesizable accelerator
store \cite{accstore}. We measure their memory bandwidth with the Intel
VTune Amplifier XE \cite{vtune}. The level of speedup for each accelerator is
calculated by dividing the execution time of one iteration of the software
benchmark on a 1.6 GHz single core by its corresponding hardware design
on a 200 MHz FPGA. The {\em Transformer} utilizes the Xilinx Spartan-3E XC3S500E
FPGA \cite{spartan3e} as a reference on-chip accelerator, which has a
comparable number of transistors, {$100\sim300$ million (500K gates)},
as eight Atom 330 cores, {$\sim 47$ million per core}
\cite{atom-spec}. The chip consists of a total number of 4,656 SLICE elements, 9,312 LUT elements
9,312, 9,312 FF elements, 768 DSP48 elements, and 20 BRAM elements. In addition, we indicate the
resource consumption percentage for each acceleration function. By
observation, the number of SLICE and BRAM elements serves as a significant
resource constraint. Therefore, we reduce the multi-dimensional
knapsack problem to a two-dimension problem. That is, we consider the
percentage of SLICE and BRAM elements for each benchmark as the two weights in
the knapsack model. The memory bandwidth utilization and level of speedup are
regarded as the two objective functions for reducing the total execution
time of memory-bound and compute-bound applications, respectively.

In the memory bandwidth-first combination, we attempt to maximize the
total bandwidth $\sum_{i=0}^{n}(BW_i \times Acc_i) $ subject to
$\sum_{i=0}^{n}(w_i \times Acc_i) $, where $Acc_i$ $i \in {1,2,3,
  ... ,n}$ denotes the different accelerator functions, $BW_i$ denotes the
bandwidth cost, and $w_i$ denotes their weight, that is the SLICE resource
demand. In the speedup-first combination selection, we focus our
efforts on maximizing the speedup metrics $\sum_{i=0}^{n}(SP_i) $, where
$SP_i , i \in {1,2,3, \ldots, n}$ denotes the speedup of each
accelerator. We attempt to maximize the two metrics by improving
the system bandwidth utilization and choosing a function with
 a significant level of speedup, which are two main approaches to improving system
performance. For example, our experiment results shows that if we
offload memory-bound benchmarks such as IDSI, SURF, Segmentation and
Jacobi to reconfigurable logics with higher memory bandwidth, we can
expect better overall performance and power efficiency. The aforementioned
heuristics are shown to be effective through extensive experimentation, which is
described further in Section \ref{sec_perf}.

\subsubsection{Dynamic Programming}

Both the two-constraint 0-1 knapsack problem and the two-constraint bounded knapsack problem can be solved
through the use of dynamic programming. 

\noindent{\em 0-1 Knapsack Problem Solution}

Assume we have $n$ acceleration functions to choose from, which are
annotated as $Acc_1$, $Acc_2$, \ldots, $Acc_n$. All $n$ items have
a two-constraint weight $a_i$, $b_i$ and a value $v_i$, $i \in [1,
  2, \ldots, n]$. We define $V[i, a, b]$ to be the current
maximum value we can obtain with a weight less than or equal to $a$
and $b$ using first $i$ accelerators. The upper limit of the two weights
are $A$ and $B$. We derive recursive equations for $V[i, a, b]$:

\begin{equation}
V[i, a, b] = V[i-1, a, b] \quad \text{if} \quad a_i > a, b_i > b,
\end{equation}
implying that if adding a new accelerator exceeds the current
two-dimensional weight limit, then we do not choose the new
accelerator and the total value will not change. As a result
\begin{equation}
\begin{array}{c}
	V[i, a, b] = max(V[i-1, a, b], V[i-1, a - a_i, b - b_i] + V[i, a, b])\\
	\text{if} \quad a_i \leq a \quad b_i \leq b.
\end{array}
\end{equation}

Algorithm \ref{alg:0-1-knapsack} describes how we solve the two-constraint 0-1
knapsack problem in order to select a set of accelerator functions. 

\begin{algorithm}[htb]
\scriptsize
\caption{ 0-1 Knapsack Problem Solution}
\label{alg:0-1-knapsack}
\begin{algorithmic}[1] 
\FOR {each $a \in [0, A]$}
	\FOR {each $b \in [0, B]$}
    	\STATE $ V[a, b] = 0; $
    \ENDFOR
\ENDFOR
\FOR {each $i \in [1, n]$ } 
    \FOR {each $a \in [A, a_i]$ }
    	\FOR {each $b \in [B, b_i]$}
    	
        	\IF {$j \geq w[i]$}
            	\STATE $V[i, a, b] = max(V[i-1, a, b], V[i-1, a - a_i, b - b_i] + v[i, a, b]) $
        	\ELSE
            	\STATE $V[i, a, b] = V[i-1, a, b]$
        	\ENDIF
        \ENDFOR
    \ENDFOR
\ENDFOR
\end{algorithmic}
\end{algorithm}

\noindent{\em Bounded Knapsack Problem}

The two-constraint bounded knapsack problem can be solved in a similar
way as the 0-1 knapsack problem. Utilizing two extra annotations, $C$, which represents the
maximum number of acceleration functions that we can request for a certain
function, and $k_i$ as the number of functions $i$ we can choose, 
 similar recursive equation can be derived for $V[i, w]$.

\begin{equation}
V[i, a, b] = V[i-1, a, b] \quad \text{if} \quad a_i > a, b_i > b
\end{equation}

and

\begin{equation}
	\begin{array}{c}
		V[i, a, b] = max(V[i-1, a, b], V[i-1, a - k_i\times a_i, b - k_i\times b_i] \\
		+ k_i\times V[i, a, b])\\
		\text{if} \quad a_i \leq a \quad b_i \leq b
	\end{array}
\end{equation}

The algorithm for the bounded knapsack problem is similar to Algorithm
\ref{alg:0-1-knapsack}, and is not elaborated further due to spacial limitations.
